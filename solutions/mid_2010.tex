\documentclass[12pt]{article}
\usepackage{xr,geometry,fancyhdr,hyperref,ifpdf,amsmath,rcs,lscape}
\usepackage{color,lastpage,longtable,Ventry,url,paunits,shortcuts,smallsec,float,multicol}
\geometry{letterpaper,top=0.75in,hmargin={0.75in,0.75in},headheight=0.75in}

\onecolumn

\pagestyle{fancy}
\lhead{Name:}
\chead{SN:} 
\rhead{\rhead{page~\thepage/\pageref{LastPage}}}

\lfoot{}
\rfoot{}
\cfoot{}

\ifpdf
    \usepackage[pdftex]{graphicx} 
    \usepackage{hyperref}
    \pdfcompresslevel=0
    \DeclareGraphicsExtensions{.pdf,.jpg,.mps,.png}
\else
    \usepackage{hyperref}
    \usepackage[dvips]{graphicx}
    \DeclareGraphicsRule{.eps.gz}{eps}{.eps.bb}{`gzip -d #1}
    \DeclareGraphicsExtensions{.eps,.eps.gz}
\fi

\graphicspath{{./figures/}}

\begin{document}

\begin{center}
  A301 2010 Mid-term,  Friday Oct. 22
\end{center}

Answer all three questions (note weights).  Show all your work, and hand-in 
all the question  pages plus
the Planck function curve on p. 5 (used for problem 3) with your radiances
marked on the plot.


\begin{enumerate}


\item[1) 10 pts] Equation (\ref{eq:vary}) allows you to find the radiance at height $z_T$
if given enough information about the atmosphere.  

\begin{equation*}
  \tag{24}
I^\uparrow (z_T,\mu) = I^\uparrow (0,\mu) \, Tr_{tot} + \int_0^{z_T} Tr(\tau_T, \tau^\prime) B(z^\prime)\, d\tau^\prime/\mu
\end{equation*}

\begin{enumerate}
\item[a) 5] Draw a  sketch of the layer, and 
label $I^\uparrow (0,\mu)$, $z_T$, $\mu$, $Tr_{tot}$, $Tr(\tau_T,\tau^\prime)$, $d\tau^\prime$,  $B(z^\prime)$,
where $z^\prime= z_T/2$ km.


\item[d) 5] Use the definition of the transmissivity $Tr(\tau_T, \tau^\prime)$ to prove that 

  \begin{equation*}
    \int_0^{z_T} Tr(\tau_T,\tau^\prime) B(z^\prime)\, d\tau^\prime/\mu = \int_0^{z_T} B(z^\prime)\, dTr^\prime
  \end{equation*}

\end{enumerate}

\newpage

\textbf{Extra space for question 1 and 2}

\newpage


\item[2) 12 pts] A 5 km thick ozone layer absorbs 30\% of the ultraviolet sunlight that passes through
it when the sun is directly overhead.

\begin{enumerate}


\item[a) 3] What is the vertical optical thickness of the layer in the ultraviolet?
(UV radiation is not reflected, only absorbed/transmitted)

\item[b) 3] What is the value of the absorptivity at 4pm, when the sun is $60^\circ$ away
from the zenith?

\item[c) 2] If the UV solar flux is 200 \un{W\,m^{-2}} for overhead sun, what is
the value of the flux at 4pm?

\item[d) 4] What is the heating rate due to UV absorption at 4pm, in degrees/day?   Assume a constant air density
of 0.5 \un{kg\,m^{-3}}  and a heat capacity of 1004 \un{J\,kg^{-1}\,K^{-1}}.


\end{enumerate}


\newpage

\item[3) 11 pts] A 10 km thick layer of an an atmosphere has constant temperature
$T_{atm}$=280 K, a pressure/density scale height of $H=8\ km$ and is filled with
a gas with a mass absorption coefficient of $k_\lambda$ = 0.2 \un{kg\,m^{-2}} 
at a wavelength of 10 \mum.  Underneath this layer is a black
surface with a temperature of 290 K. The atmosphere is in hydrostatic
equilibrium, the gas has a constant mixing ratio and a density 
at the surface of $\rho_0 = 10^{-3}$ \un{kg\,m^{-3}}. Find:

\begin{enumerate}
\item[a) 3]  The vertical optical depth of the layer
\item[b) 2] The layer transmission for radiance going straight up.
\item[c) 4] The radiance, in \un{W\,m^{-2}\,sr^{-1}} 
at the top of the layer due to the surface and atmosphere.
\item[d) 2] The brightness temperature $T_b$ (K) that  a satellite would measure
at $\lambda$=10 \mum if there were no absorption/emission above 10 km.

\end{enumerate}


\end{enumerate}



\hspace{-1.5in}
\includegraphics[width=9in]{longwave.pdf}


\begin{center}
  Useful equations
\end{center}

\begin{multicols}{2}


\begin{eqnarray}
  \label{eq:nu}
  \nu &=& c / \lambda\\
   E &=& h \nu
\end{eqnarray}

\begin{equation}
  \label{eq:omega}
  d \omega = \sin \theta d\theta d\phi = -d\mu d\phi \approx \frac{ A}{r^2} 
\end{equation}

\begin{equation}
  \label{eq:cos}
  S = S_0 \cos(\theta)
\end{equation}

\begin{equation}
  \label{eq:conservation}
  a_\lambda + r_\lambda + Tr_\lambda = 1
\end{equation}

\begin{equation}
  \label{eq:intensity}
  I_\lambda = \frac{\Delta F}{\Delta \omega \Delta \lambda}
\end{equation}

\begin{equation}
  \label{eq:flux_int}
  F = \int_{2 \pi} I \cos \theta \sin \theta d \theta d \phi
\end{equation}

\begin{equation}
  \label{planck}
B_\lambda(T)  = \frac{h c^2}{\lambda^5 \left [ \exp (h c/(\lambda k_B T )) -1 \right ] }
\end{equation}


\begin{equation}
  \label{eq:pi}
  F_{\lambda\,BB} = \pi B_\lambda
\end{equation}


\begin{equation}
  \label{eq:stefan}
  F_{BB}=\sigma T^4
\end{equation}

\noindent
Heating rate equation
\begin{equation}
  \label{eq:fluxdiv}
  \frac{dT}{dt} = \frac{-1}{\rho c_{pd}} \frac{dF_n}{dz}
\end{equation}


\begin{equation}
  \label{eq:taylor}
  F_\lambda(T) \approx F_{\lambda\, 0} + \left .\frac{dF_\lambda}{dT}  \right |_{T_0,\lambda} \!\!\! (T - T_0) + \ldots
\end{equation}

\begin{equation}
  \label{eq:exp}
  \exp(x) = 1 + x +  \frac{x^2}{2} + \frac{x^3}{3!} + \ldots
\end{equation}

\vspace{0.5in}



$~$



\noindent
Beer's law for extinction:
\begin{eqnarray}
  \label{eq:extinct}
\frac{dI_\lambda}{I_\lambda}  & = & - \kappa_{\lambda\, s} \rho_{g} ds - 
                    \kappa_{\lambda\,a } \rho_{g} ds \nonumber\\
        &=& - \kappa_{\lambda e} \rho_{g} ds = \kappa_{\lambda e} \rho_{g} dz/\mu
\end{eqnarray}
(assuming $\rho_{a}$=$\rho_{s}$=$\rho_{g}$).

\noindent
Beer's law integrated:

\begin{equation}
  \label{eq:binteg}
  I= I_0 \exp (- \tau/\mu)
\end{equation}


$~$

\noindent
Hydrostatic equation:

\begin{equation}
  \label{eq:hydro}
  dp = -\rho_{air}\, g\, dz
\end{equation}

$~$

Hydrostatic equation integrated:

\begin{equation}
  \label{eq:hydroint}
p = p_0 \exp(-z/H)
\end{equation}


\noindent
Equation of state

\begin{equation}
  \label{eq:state}
  p = R_d \rho_{air} T
\end{equation}


$~$

\noindent
vertical optical thickness:

\begin{equation}
  \label{eq:tauThick}
  d \tau =  \kappa_\lambda \rho_g dz = \kappa_\lambda r_{mix} \rho_{air} dz = \beta_a dz
\end{equation}

\noindent
integrate vertical optical thickness:

\begin{equation}
  \label{eq:tauup}
  \tau(z_1, z_2 ) = \int_{z_1}^{z_{2}} k_\lambda r_{mix} \rho_{air}\, dz^\prime
\end{equation}


\noindent
Transmission function for upwelling radiation

\begin{eqnarray}
Tr(\tau_1,\tau_2) &=& \exp ( - (\tau_1 - \tau_2)/\mu ) \nonumber\\
\end{eqnarray}

$~$


$~$

\noindent
Schwarzchild's equation for an absorbing/emitting gas

\begin{equation}
  \label{eq:schwarz}
  dI = -I\, d\tau/\mu + B_{\lambda}(z) d \tau /\mu
\end{equation}

$~$

\noindent
For an isothermal atmosphere:

\begin{equation}
\label{eq:isothermal}
I_\uparrow = I_0 \exp( - \tau_T /\mu) + (1 - \exp( - \tau_T /\mu)) B_\lambda(T)
\end{equation}


\noindent
Temperature varying with height:

\begin{gather}
  \label{eq:topa}
  I^\uparrow (z_T,\mu) = \nonumber \\
I^\uparrow (0,\mu) \, Tr_{tot} + \int_0^{z_T} Tr^\prime B(z^\prime)\, d\tau^\prime/\mu \label{eq:vary}\\
I^\uparrow (0,\mu) \, Tr_{tot} + \int_0^{z_T} B(z^\prime)\, dTr^\prime \label{eq:trans}
\end{gather}



\rule{3cm}{.1mm}



$~$

\noindent
Useful constants:

$~$

$c_{pd}=1004$ \un{J\, kg^{-1}}, 

$\sigma=5.67 \times  10^{-8}$ \un{W\,m^{-2}\,K^{-4}}

$k_b = 1.381  \times 10^{-23}$ \un{J\,K^{-1}} 

$c=3 \times 10^{8}$ \un{m\,s^{-1}}

$h=6.626 \times 10^{-34}$ \un{J\,s}

$\pi \approx 3.14159$

$R_d=287 \un{J\,kg^{-1}\,K^{-1}}$

Solar radius=$7 \times 10^8$ m

Earth-sun distance =$1.5 \times 10^{11}$ m

\end{multicols}

\end{document}
